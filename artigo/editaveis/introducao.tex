\chapter[Introdução]{Introdução}
\section[Contextualização]{Contextualização}

A popularização da internet foi um fenômeno que possibilitou e facilitou a criação e o compartilhamento de conteúdo, além de ter criado uma enorme abertura para a exposição de opiniões. Junto com a liberdade de expressão, existe o risco da manifestação de discursos ofensivos, incitação à violência, discriminação contra grupos em virtude da etnia, religião, orientação sexual, entre outros. Assim, é definido o "discurso de ódio". Em redes sociais, é algo bastante recorrente, onde é possível observar o exercício abusivo da liberdade de expressão, com as pessoas assumindo uma postura mais ativa, potencializado pela alta velocidade de propagação e aparente possibilidade de anonimato (ROTHENBURG, STROPPA, 2015). 

O projeto Empurrando Juntos foi criado com o intuito de ser uma rede social que objetiva minimizar os efeitos da polarização, das bolhas de opinião e das manipulações que existem nas redes sociais mais populares da atualidade. Através da horizontalidade de comentários, de elementos de gamificação que permitem um maior nivelamento entre os comentários, e de uma lógica de audiência operada por Inteligências artificiais (PARRA, HENRIQUE, 2017). Dentre essas inteligências artificiais, destaca-se aqui a responsável pelo gerenciamento de comentários e conversas dentro da plataforma que, dado o contexto em que a plataforma se insere, se faz necessário.

\section[Problematização]{Problematização}


A plataforma Empurrando Juntos busca discutir diferentes assuntos, políticos e sociais, com o objetivo de dar voz à quem precisa e de construir uma comunidade mais crítica e mais aberta a opiniões diferentes. Por ser classificado como um “Site de Rede social”, é considerado um “local público” (Rebs, 2017), mas com algumas diferenças do local público físico, como a replicabilidade, a persistência e a buscabilidade de informações, uma vez que a mesma sempre fica disponível. Dada a necessidade da plataforma, o foco na moderação é um dos requisitos para o seu sucesso, e a moderação automatizada é a opção ideal para sistemas com grandes quantidades de entrada de informação e interação.

O principal canal de informação da plataforma EJ se dá a partir das conversas e dos comentários. A fim de promover um ambiente de discussão mais saudável, eliminando xingamentos, ofensas gratuitas e o discurso de ódio, uma moderação de comentários é realizada. Esta atividade é executada manualmente por membros do projeto, que devem ler os comentários individualmente para poder aprovar a sua exibição na plataforma.

O atual trabalho propõe a automatizar o processo de moderação o máximo possível, de modo que, nem todos os comentários precisem ser submetidos à moderação manual, sendo estes, automaticamente classificados como discurso comum e exibidos, ou discurso de ódio, e deletados. Como ferramenta de apoio da plataforma, o moderador automático não promove a substituição completa da moderação manual, apenas a diminuição da carga de trabalho destinada ao processo.

\section[Objetivos]{Objetivos}

Este trabalho possui como objetivo geral facilitar a identificação de comentários com o linguajar inapropriado e conteúdo incompreensível para aplicações web, mais especificamente no projeto "Empurrando Juntos", utilizando de técnicas e ferramentas de processamento de linguagem natural e aprendizado de máquina.

Visando atingir o objetivo geral, alguns objetivos específicos foram traçados:

\begin{itemize}
\item Selecionar conjunto de dados de treinamento
\item Avaliar modelos de classificação
\item Otimizar hiperparâmetros dos modelos de classificação
\item Realizar integração com a plataforma Empurrando Juntos
\item Implantar a retroalimentação do modelos de classificação
\end{itemize}

\section[Organização dos capítulos]{Organização dos capítulos}

O atual trabalho está organizado em capítulos de acordo com a lista abaixo:

\begin{itemize}
\item \textbf{Introdução:} atual capítulo, onde são apresentadas a contextualização, as motivações e os objetivos. 
\item \textbf{Fundamentação Teórica:} capítulo dedicado a fornecer o embasamento teórico necessário para a realização e o entendimento deste trabalho.
\item \textbf{Ferramentas:} apresenta as ferramentas que deram suporte necessário para a execução do trabalho, tanto na parte experimental quanto na seu desenvolvimento.
\item \textbf{Metodologia:} detalha a abordagem e os procedimentos seguidos na execução deste trabalho.
\item \textbf{Resultados Parciais:} apresenta os resultados obtidos através da coleta das métricas dos experimentos realizados.
\item \textbf{Considerações finais:} trata-se de um capítulo dedicado à uma análise interpretativa do trabalho como um todo.
\end{itemize}
