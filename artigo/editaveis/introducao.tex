\chapter*[Introdução]{Introdução}
\addcontentsline{toc}{chapter}{Introdução}

A popularização da internet foi um fenômeno que possibilitou e facilitou a criação e o compartilhamento de conteúdo, além de ter criado uma enorme abertura para a exposição de opiniões. Junto com a liberdade de expressão, existe o risco da manifestação de discursos ofensivos, incitação à violência, discriminação contra grupos em virtude da etnia, religião, orientação sexual, entre outros. Assim, é definido o "discurso de ódio".

Em redes sociais, é algo bastante recorrente, onde é possível observar o exercício abusivo da liberdade de expressão, com as pessoas assumindo uma postura mais ativa, potencializado pela alta velocidade de propagação e aparente possibilidade de anonimato (ROTHENBURG, STROPPA, 2015).

\begin{comment}
## TODO Formatar a referência acima http://coral.ufsm.br/congressodireito/anais/2015/6-21.pdf
\end{comment}

Este trabalho tem como objetivo facilitar a identificação de discursos de ódio em comentários para aplicações web, mais especificamente no projeto do "Empurrando Juntos". Para isso, serão utilizadas técnicas e ferramentas de processamento de linguagem natural e aprendizado de máquina.
