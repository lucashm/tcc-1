\chapter[Ferramentas]{Ferramentas}

\section{Seleção do Dataset}

O processo de aprendizado de máquina é composto por uma série de etapas, dentre elas temos a seleção de um conjunto de dados (Dataset) que servirá de entrada para a realização do "treinamento" da máquina.

\subsection{Dataset de análise de sentimentos}
Como protótipo inicial, foi escolhido um dataset provido pelo graduando da Universidade federal da Paraíba, Paulo Emílio. O dataset foi obtido através da API do Twitter, onde tweets contendo um rosto feliz (':)', ':-)') foram classificados como positivos, e tweets com um rosto triste (':(', ':-('), classificados como negativo. Contudo, percebeu-se que muitos comentários positivos estavam "contaminando" o dataset, visto que muitos deles continham conteúdo irônico, que nem sempre indicava uma frase positiva. Desse modo, decidiu-se determinar o sentimento de cada tweet manualmente, até uma quantidade adequada.

\subsection{Dataset de discurso de ódio}
Um dos grandes desafios deste projeto é encontrar ou constuir um dataset substancial de discurso de ódio em português. O dataset encontrado que mais se aproximava do desejado foi o OffComBR (PELLE, 2017). Ele é composto por mais de 1000 exemplares de texto e classificados com 'yes' para ofensivo, e 'no' para não ofensivo.

\begin{comment}
## TODO Formatar a referência acima

@inproceedings{Pelle2017,
title={Offensive Comments in the Brazilian Web: a dataset and baseline results},
author={Rogers P. de Pelle and Viviane P. Moreira},
booktitle={6th Brazilian Workshop on Social Network Analysis and Mining (BraSNAM)},
year={2017},
note={to appear}
}

\end{comment}




\section{Filtragem dos dados}

\section{Aplicação de Tags}

\subsection{Semi automatização}

\section{Descrição da infraestrutura}
