\chapter[Ferramentas]{Ferramentas}

\section{Seleção do Dataset}

\subsection{Dataset de análise de sentimentos}
Como protótipo inicial, foi escolhido um dataset provido pelo graduando da Universidade federal da Paraíba, Paulo Emílio. O dataset foi obtido através da API do Twitter, onde tweets contendo um rosto feliz (':)', ':-)') foram classificados como positivos, e tweets com um rosto triste (':(', ':-('), classificados como negativo. Contudo, percebeu-se que muitos comentários positivos estavam "contaminando" o dataset, visto que muitos deles continham conteúdo irônico, que nem sempre indicava uma frase positiva. Desse modo, decidiu-se determinar o sentimento de cada tweet manualmente, até uma quantidade adequada.


\section{Filtragem dos dados}

\section{Aplicação de Tags}

\subsection{Semi automatização}

\section{Descrição da infraestrutura}

