\chapter{Considerações Finais}

A plataforma Empurrando Juntos foi apresentada como um ambiente inovador de discussão e engajamento social e que busca trazer um ambiente que favoreça um debate mais produtivo, partindo de um ponto de vista completamente diferente das redes sociais atuais. Com um enfoque em tópicos de cunho social, a plataforma pode tratar de assuntos delicados, como preconceito, homofobia e racismo. Existe, portanto, a necessidade de se filtrar o conteúdo, principalmente pelo objetivo de manter o decoro do ambiente, impedir o assédio moral e disseminação do discurso de ódio. Com base nisso, este projeto propõe uma solução que automatiza, mesmo que parcialmente, a moderação de comentários da plataforma.

Como continuação deste trabalho, no TCC 2, a ideia é realização de novas iterações do ciclo de vida do CRISP-DM, assim como a construção de um comitê composto por múltiplos modelos dentre os que foram estudados até então. Suas classificações serão utilizadas e ponderadas com pesos pré-definidos a partir das métricas colhidas, para alcançarmos uma classificação final, que, por fim, será utilizada na moderação dos comentários da plataforma EJ.
	
O processo de implantação do classificador na plataforma será realizada mediante ao estudo de sua arquitetura e aplicação de padrões de projeto. Também faz parte do planejamento, a utilização das comentários com as avaliações finais do moderador como insumo que irá retroalimentar os modelos. Este processo deverá ser realizado iterativamente, assim como novas etapas de refinamento dos hiperparâmetros.

Observe abaixo as atividades previstas para a realização do TCC 2.

\begin{table}[htbp]
\centering
\begin{tabular}{@{}ll@{}}
\toprule
\textbf{Prazo} & \textbf{Atividade} \\ \midrule
Maio & Finalizar correções do TCC1 \\
Maio & Revisão mais profunda da literatura de PLN e AM \\
Maio & Estudar a arquitetura da Plataforma EJ \\
Maio & Desenvolver comitê de máquina, avaliar e refinar parâmetros \\
Junho & Realizar a implantação do algoritmo na Plataforma EJ \\
Junho & Avaliar o desempenho e realizar melhorias no moderador \\
Julho & Finalizar documentação do TCC 2 \\
Julho & Apresentar TCC 2 \\ \bottomrule
\end{tabular}
\caption{Cronograma para o TCC 2}
\label{tab:crono-tcc2}
\end{table}

